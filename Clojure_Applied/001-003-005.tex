\section{ソートと重複排除}

最も基本的なソート機能は \texttt{sort} で、デフォルトのコンパレータまたは必要に応じてカスタムのコンパレータでソートすることができます。例えば、これはアルファベット順で最初の5つの惑星名を取得します。


\begin{lstlisting}[numbers=none]
(take 5 (sort (map (:name planets))))
\end{lstlisting}

この例では、惑星の名前を取得し、その名前をソートしています。しかし、しばしば、元の実体を惑星名で並べ替えたいことがある。つまり、値を取り出すのではなく、各要素に適用される関数でソートしたいのです。これは\texttt{sort-by}で実現できる。


\begin{lstlisting}[numbers=none]
(take 5 (sort-by :name planets))
\end{lstlisting}

\texttt{sort} と \texttt{sort-by} はどちらも出力の実体化を必要とするので、遅延シーケンスを返すことはありません。どちらもトランスデューサのバージョンはありません。

最小の\texttt{n}個の惑星を取り出すには、まず体積の大きい順にソートし、それから最初のn個を取り出す必要があります。



\begin{lstlisting}[numbers=none]
(defn smallest-n
  [planets n]
  (->> planets
       (sort-by :volume)
       (take n)))
\end{lstlisting}

非セットコレクションの中には、重複を含むものがあります。これらは\texttt{distinct}で除去できますが、これはこれまでに見た要素を追跡する必要があり、大きな入力コレクションではメモリの問題になる可能性があります。より限定された機能でこの問題を回避する代替案は、\texttt{dedupe}であり、後続の重複する値を削除する。\texttt{dedupe}関数は前の要素をメモリ上に保持するだけなので、大きな入力に対してより安全に使用できる。


