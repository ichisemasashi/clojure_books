\section{まとめ}

これで、ドメインモデルの内部と外部の両方でコレクションを使用して、エンティティと値の両方を収集する方法について完全に理解しました。Clojureアプリケーションのデータのほとんどは、ここで説明したコレクション以外から構築されていません。時折、特殊な考慮事項やパフォーマンスを最大化するために、独自のコレクションを構築することが有用であることが分かるでしょう。

第4章「状態、アイデンティティ、および変更」で状態をどのように管理することを期待するかについて、段階を踏んでいます。この章で説明した概念と同様に、状態管理は不変の値と純粋な変換関数の基礎に大きく依存しています。

しかし、まずはコレクションと関数に関する知識を活かして、データを処理する能力をどのように拡張するかに焦点を当てます。これまでは、主にコレクション・レベルで、単一の値やエンティティを変更してきました。次に、範囲を広げてシーケンスについて説明します。

シーケンスとは、リストやベクターなどのコレクションをシーケンシャルなデータ構造のように扱えるようにするための一般化表現です。Clojureのデータ変換機能のほとんどは、特定のコレクションに縛られることなく、このより一般的な抽象化の上に構築されています。Clojureのデータ変換関数は、強力で再利用可能なClojureアプリケーションを書くための重要な部分です。
