\section{まとめ}

ClojureコレクションはClojureデータの不変のベースを提供し、シーケンスはコレクションと他の順次トラバース可能なデータソースの両方の上に重要な抽象化を提供します。シーケンス関数とトランスデューサの両方を使ったシーケンシャルデータの最も一般的な処理方法を示しました。

トランスデューサーは、シーケンス処理モデルを、ソースの反復処理、変換、出力処理に分割し、それぞれを独立して変更できるようにすることで、より良いパフォーマンスとより多くの再利用性を獲得しています。入力ソースにトランスデューサを適用する3つの一般的な方法として、\texttt{sequence}、\texttt{into}、\texttt{transduce}の使い方を見ました。今後の章では、これらと同じトランスデューサー関数を \texttt{core.async} チャンネルに適用する方法も紹介します。

さて、ドメインをモデル化し、ドメインエンティティをコレクションにグループ化し、それらを処理したところで、スレッドと時間をまたぐ状態の連携を開始する方法を検討する必要があります。