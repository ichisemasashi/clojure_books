\section{まとめ}

コンポーネントは、アプリケーションの実際の機能を確立する、より大きなコードの単位を構築するための手段です。コンポーネントは構造を提供し、チームが作業を分担するために使用できるコードの有意義なサブユニットを作成します。

まず、コンポーネントの外部API(関数、非同期呼び出し、チャンネルによるイベントストリーム)を設計する方法と、 core.async を使用してこれらのコンポーネントを接続する方法について見ていきました。

また、各コンポーネントの設定データ、依存関係、内部状態、コンポーネントのライフサイクルを考慮した実装方法についても見てきました。

次に、システムを組み立てるための全体像を探ります。アプリケーションの設定データをどのように管理し、コンポーネントに提供するか、コンポーネントをどのようにインスタンス化して接続するか、システムのエントリポイントをどのように提供するか、について見ていきます。