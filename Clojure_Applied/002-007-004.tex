\section{システム構成}

システム構成には、システム属性、環境ごとの情報、開発者専用の情報など、いくつかの種類の設定が含まれています。システム属性とは、アプリケーションの動作に影響を与えるフラグやその他の設定のことで、機能のオン・オフや、いつか変更する必要のあるマジックナンバーを外部化することができます。環境ごとの情報は、開発、品質保証、本番など、アプリケーションの展開先ごとに変化します。そして最後に、開発者専用の設定は、開発者が作業中に自分のマシン上で環境を微調整することを可能にします。

これらの設定のうち、ソースコントロールにチェックできるのはシステム属性のみです。環境ごとの設定は、アプリケーションの外側の環境で設定する必要があります。dev- onlyの設定は、個々の開発時にローカルにのみ設定されるべきものです。

起動時に、これらすべての種類の値をまとめて、システム設定の一貫したビューにロードする必要があります。ここでは、複数のソースから取得した値に対する一貫したインターフェースを得るための一つの方法として、Environライブラリについて見ていきます。もう少し深い解決策としては、Immuconf ライブラリを利用することにします。

\subsection{Environ}

Environライブラリは、Leiningenプロファイル、環境変数、およびJavaシステムプロパティから取得した設定値を一元的に表示するものです。\texttt{:rule-set}(使用する知識エンジンのルールセット)、\texttt{:feed1-user}(ソーシャルメディアのフィードのユーザー名)、\texttt{:verbose}(開発時のデバッグフラグ)の3つのシステム構成プロパティを考えてみましょう。

開発時には、ローカルビルドで\texttt{rule-set}と\texttt{feed1-user}を設定したいと思うかもしれません。Environでこの設定を行うには、まずLeiningenの\texttt{project.clj}を更新して、Environを依存関係として(コードに)、lein-environプラグインを(ビルドに)追加する必要があります。


\begin{lstlisting}[numbers=none]
:dependencies [[environ "1.0.0"]]
:plugins [[lein-environ "1.0.0"]]
\end{lstlisting}

システム構成プロパティを設定する最初の場所は、\texttt{project.clj} の中で Leiningen プロファイルの一部として直接設定することができます。プロファイルを使用すると、プロジェクトのビルドにオプションで含まれるプロジェクト設定のバンドルを作成することができます。プロファイルの一般的な使い方の1つは、環境固有のプロジェクト環境を作成することです。Leiningen では、プロファイル名とプロファイル固有のプロジェクト設定のマップを含む \texttt{:profiles} キーを使用します。

\begin{lstlisting}[numbers=none]
:profiles {:dev  { ,,, }
           :qa   { ,,, }
           :prod { ,,, }}
\end{lstlisting}

Environライブラリは、プロジェクト設定にある\texttt{:env}キーからシステム設定を読み込むことを想定しています。

\begin{lstlisting}[numbers=none]
:profiles {:dev {:env {:rule-set "basic"}}
           :prod {:env {:rule-set "advanced"}}}
\end{lstlisting}

Leiningenでは \texttt{:user} と \texttt{:dev} のプロファイルはよく知られていて、デフォルトでオンになっているので、\texttt{lein repl}でいつも通りREPLを起動すれば、devの設定が利用できるようになります。

\begin{lstlisting}[numbers=none]
user=> (require '[environ.core :refer (env)])
nil
user=> (env :rule-set)
"basic"
\end{lstlisting}

しかし、\texttt{project.clj}ファイルは、通常、ソースコントロールシステムで追跡されます。\texttt{:rule-set}のような環境設定は問題ありませんが、別の構成設定(データベースのユーザー名とパスワードなど)はおそらく受け入れられません。その場合、開発者固有の設定として、ソース・コントロールの外にある別の Leiningen \texttt{profiles.clj} に設定を保存することをお勧めします。

この例では、\texttt{project.clj} から \texttt{:profiles} を削除し、代わりに \texttt{profiles.clj} の中に \texttt{:profiles} キーの値を入れることで、この例を変更することができます。


\begin{lstlisting}[numbers=none]
{:dev {:env {:rule-set "basic"}}
 :prod {:env {:rule-set "advanced"}}}
\end{lstlisting}

そして、\texttt{profiles.clj}は、ソースコントロールシステムによって無視されるようにマークされるべきです。このようにして、すべての開発者は、機密の設定を公開することなく、自分の環境を指定することができます。

本番環境にデプロイするときは、もう Leiningen を使うことはありません。代わりに、JVMシステムプロパティまたはシステム環境変数によって、これらのプロパティを提供することができます。

