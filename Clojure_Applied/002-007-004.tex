\section{システム構成}

システム構成には、システム属性、環境ごとの情報、開発者専用の情報など、いくつかの種類の設定が含まれています。システム属性とは、アプリケーションの動作に影響を与えるフラグやその他の設定のことで、機能のオン・オフや、いつか変更する必要のあるマジックナンバーを外部化することができます。環境ごとの情報は、開発、品質保証、本番など、アプリケーションの展開先ごとに変化します。そして最後に、開発者専用の設定は、開発者が作業中に自分のマシン上で環境を微調整することを可能にします。

これらの設定のうち、ソースコントロールにチェックできるのはシステム属性のみです。環境ごとの設定は、アプリケーションの外側の環境で設定する必要があります。dev- onlyの設定は、個々の開発時にローカルにのみ設定されるべきものです。

起動時に、これらすべての種類の値をまとめて、システム設定の一貫したビューにロードする必要があります。ここでは、複数のソースから取得した値に対する一貫したインターフェースを得るための一つの方法として、Environライブラリについて見ていきます。もう少し深い解決策としては、Immuconf ライブラリを利用することにします。

\subsection{Environ}







