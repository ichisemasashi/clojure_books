\section{キューとワーカー}

多くのプログラムは、その全体または一部をタスク処理系として見ることができます。タスクとは、通常、外部からの要求に対応する作業の単位です。ウェブアプリケーションは、ウェブページを作成するためのリクエストを受け取ります。ウェブサービスは、APIコールを処理するリクエストを受け取ります。バッチプログラムはディスクやデータベースからファイルを読み込んで、それぞれを適切に処理する。これらの一般的なパターンはすべて、ワーカーのプールに委託された作業のキューとしてモデル化することができます。

キューは、タスクの順序付けと保持を行い、ワークが到着する場所と処理される場所を切り離します。ワーカープールでは、異なる特性を持つワーカーのプールを作成し、並行処理や、作業の管理・監視に使用するポリシーを制御することができます。その制御により、自由に使えるハードウェアをフルに活用することができるのです。

Clojureはキューとワーカーのためのいくつかのツールを提供しますが、Javaですでに利用可能な高品質のツールを再発明することも避けます。Clojureで既に見てきた部分から、キューとワーカープールをどのように作成できるか考えてみましょう。

\subsection{組み立てが必要}

第1章Model Your Domainでは、Clojureの永続キューを使用して、FIFOデータに対してリストやベクターよりも効率的なアクセスを提供しました。しかし、これはシングルスレッドのコンテキストで行われました。永続的なキューでは、キューを変更するたびに、更新されたバージョンが返されます。もし複数のスレッドがキューを共有するならば、それら全てが同じインスタンスを共有する必要があります。そのため、(atom や ref などの) 状態管理構造体か、状態管理型のキュー実装のいずれかが必要になります。

1つのオプションは、キューの両端が安定したアイデンティティを維持するように、Clojureのアトムまたは参照に永続的なキューをラップすることです。アトムでこれを行おうとすると、アトムの\texttt{swap!}関数がアトムの新しい値(私たちの場合はキュー)を返すだけで、ポップした値を返さないことに気づきます。このため、この種のキューからステートフルな方法でアイテムを取り出すことは困難です。

refオプションはより有望に見えます。このように実装できるだろう。


\begin{lstlisting}[numbers=none]
(defn queue
  "新しいステートフルキューの作成"
  []
  (ref clojure.lang.PersistentQueue/EMPTY))

(defn enq
  "qのitemを待ち受ける"
  [q item]
  (dosync
    (alter q conj item)))

(defn deq
  "qからitemを取り出す(なければnil)。"
  [q]
  (dosync
    (let [item (peek @q)]
      (alter q pop)
      item)))
\end{lstlisting}

しかし、このキューはブロックしない! 通常、キューが空でデータの到着を待っているときにコンシューマが \texttt{deq} でブロックすることを望みますが、この実装では \texttt{nil} を返すだけで、コンシューマが繰り返しポーリングすることを要求されます。この理由から、Clojureの永続的なキューは、通常、スレッド間で仕事のキューを管理するための良いツールではありません。

代わりに、キューとワーカーに対するJavaのサポートに注目する必要があります。これは、Javaライブラリが多種多様な動作に対して強力なサポートを持っている分野です。

\subsection{Javaキュー}

キューとワーカーをサポートするJavaクラスのほとんどは、\texttt{java.util.concurrent}パッケージで見つけることができます。Javaは多くのブロッキングキュー実装(すべて\texttt{java.util.concurrent.BlockingQueue}の実装)を提供し、それらはClojureから簡単に使用することができます。

Java のキュー実装の主な違いの1つは、データのバッファリング方法です。例えば、\texttt{LinkedBlockingQueue}はオプションで境界付きバッファを提供し、\texttt{ArrayBlockingQueue}は境界付きバッファを提供し、\texttt{SynchronousQueue}はバッファを全く提供しません。\texttt{LinkedTransferQueue} は、\texttt{SynchronousQueue} のハンドオフ機能と、オプションで制限されたバッファを組み合わせています。

これまで述べてきたすべてのキューは FIFO 順で値を提供しますが、Java は項目を並べ替えるキューも 2 つ提供します。\texttt{PriorityBlockingQueue} は、優先度の高い項目をキューの先頭にバブリングします。\texttt{DelayQueue}は、遅延のあるメッセージを受け取り、遅延の期限が切れたものだけを利用できるようにします。

Bounded buffer queue は、producer が満杯のバッファに遭遇したときに、カスタマイズする機会も提供します。Java のブロッキングキュー API では,ブロッキング,タイムドブロッキング,特別な値の返送,例外の発生が可能です.

\texttt{put} や \texttt{take} などの \texttt{BlockingQueue} メソッドは、通常の Java interop メソッド呼び出しで呼び出すことができます。それでは、いくつかのメッセージをキューにプッシュしてみましょう。

\begin{lstlisting}[numbers=none]
(ns ch5.jqueue
  (:import [java.util.concurrent LinkedBlockingQueue]))

(defn pusher [q n]
  (loop [i 0]
    (when (< i n)
      (.put q i)
      (recur (inc i))))
    (.put q :END))

(defn popper [q]
  (loop [items []]
    (let [item (.take q)]
      (if (= item :END)
          items
          (recur (conj items item))))))

(defn flow [n]
  (let [q (LinkedBlockingQueue.)
        consumer (future (popper q))
        begin (System/currentTimeMillis)
        producer (future (pusher q n))
        received @consumer
        end (System/currentTimeMillis)]
    (println "Received:" (count received) "in"
             (- end begin) "ms")))
\end{lstlisting}

\texttt{pusher}関数は、キューに\texttt{n}個の数値をプッシュし、最後に完了を知らせる\texttt{:END}メッセージを送ります。\texttt{popper}関数は同じキューから\texttt{:END}メッセージを受け取るまでメッセージを引き離します。これらの関数は両方ともバックグラウンドのスレッドプールで実行される \texttt{futures} で実行されます。

しかし、どのスレッドがこれらの\texttt{future}を実行するかは制御できません。Clojureのfutureとagentは非同期実行のための比較的簡単なAPIを提供しますが、監視と制御が多少損なわれます。その代わりに、多くのスレッド上で作業を実行するためのJavaのビルトインサポートを使用することができます。

\subsection{スレッドの作成}

Javaには、スレッドのファクトリー(\texttt{ThreadFactory})と、キューとワーカープールの組み合わせ(\texttt{ExecutorService})を表現するインタフェースが用意されています。

このように、プロセッサ数に応じた大きさの計算スレッドの固定プールを作成することができます。


\begin{lstlisting}[numbers=none]
(import '[java.util.concurrent Executors])

(def processors (.availableProcessors (Runtime/getRuntime)))

(defonce executor (Executors/newFixedThreadPool processors))

(defn submit-task [^Runnable task]
  (.submit executor task))
\end{lstlisting}

Javaでは、\texttt{Runnable}または\texttt{Callable}インタフェースを用いて実行可能なタスクを表現します。役に立つことに、すべてのClojure無引数関数はこれらのインターフェイスを実装しています。タスク(任意のClojure関数)は、呼び出すために\texttt{ExecutorService}に渡すことができます。したがって、リクエストのストリームをタップして、それらを実行のためのタスクとして送信するのは簡単です。

Java 5で追加されたJavaエグゼキュータは、当時一般的だった4~8コアのマシンで粗視化されたタスク並列をサポートするように設計されています。しかし、マシンのコアが増えるにつれ、1つのキューで待つことによる競合が、キューからアイテムを取得するワーカーのボトルネックとなりました。

この問題に対処し、他の計算パターンを利用するために、Java 7 では \texttt{fork/join} という新しいフレームワークが導入されました。\texttt{fork/join} は、より小さな粒度の細かい計算タスク、再帰的な計算、より多くのコアをサポートするように設計され、調整されています。\texttt{fork/join} は多数のワーカーキューを使用し、それらのキューが互いに仕事を「盗む」ことを可能にします。つまり、あるキューがやるべき仕事を使い果たした場合、別のキューの後ろからタスクを取り出し、キュー間の仕事のバランスを自動的に調整します。

\texttt{java.util.concurrent.ForkJoinPool} クラスは、Java のフォーク/ジョイン実装の主要なエントリポイントです。\texttt{ForkJoinPool}を構築すると、それは\texttt{ExecutorService}でもあり、同じようにタスクを投入することができます。しかし、Clojureは、Clojure開発者にとってより自然な方法で\texttt{fork/join}を活用するフレームワークを提供します。次に、そのフレームワークをいつ、どのように使用するかを見ていきます。